\documentclass[a4paper]{article}
\pdfoutput=1

\usepackage{libertine}
\usepackage{geometry}
\usepackage[round]{natbib}                      % For bibliography style
\renewcommand{\sfdefault}{lmss}  % Computer Modern Sans
\renewcommand{\bfdefault}{bx}

\usepackage{hyperref}
\usepackage{xcolor}
\definecolor{dark-blue}{rgb}{0,0,0.7}
\hypersetup{
    colorlinks, linkcolor={dark-blue},
    citecolor={dark-blue}, urlcolor={dark-blue}
}

\usepackage{titlesec}
\titleformat{\section}{\bfseries\sffamily\large\color{black}}{\thesection}{1em}{}
\titleformat{\subsection}{\bfseries\sffamily\color{black}}{\thesubsection}{0.5em}{}
\titleformat{\subsubsection}{\bfseries\sffamily\color{black}}{\thesubsubsection}{0.4em}{}

\usepackage[acronym,smallcaps,nowarn,section,nonumberlist]{glossaries}
\glsdisablehyper{}

\usepackage{scalefnt,letltxmacro}
\LetLtxMacro{\oldtextsc}{\textsc}
\renewcommand{\textsc}[1]{\oldtextsc{\scalefont{1.10}#1}}

\newacronym{MDP}{mdp}{Markov decision process}
\newacronym{RL}{rl}{Reinforcement learning}
%----------------------------------------------------------------------------


\usepackage[alpha]{mdpn}   % Load the MDP notation style


%----------------------------------------------------------------------------
%   TITLE SECTION
%----------------------------------------------------------------------------
\author{Billy Okal\\\\
University of Freiburg}

\title{\bf \sffamily Example Usage of MDPNv1 Notation}
\date{}



\begin{document}

\maketitle


\section{Introduction}
\label{sec:intro}
Many \gls{RL} research papers contain paragraphs that define \gls{MDP} and related concepts.
These paragraphs take up space that could otherwise be used to present more useful/new content.
In this paper we specify a notation for \gls{MDP} that can be used by other papers. Declaring the use this notation using a single sentence can replace several paragraphs of notational specifications in other papers.
Importantly, the notation that we define is a common foundation that appears in many \gls{RL} and related papers, and is not meant to be a complete notation for an entire paper.

\section{MDPs using MDPNv1 notation}
\label{sec:mdps}
A \gls{MDP} is denoted by a tuple, $\MDP$, where;
%
\begin{enumerate}
    \item We use $\step$ to denotes the time step, where $\NZ$ denotes the natural numbers {\em including zero}.

    \item $\sset$ is the set of possible states that the agent can be in, and is called the {\em state set}. The state of the environment at time $t$ is a random variable that we denote by $\st{t}$. We will typically use $s$ to denote an element of the state set. $\Pol$

    \item Similarly, $\aset$ is the set of possible actions the agent can perform. The action at time $t$ is denoted by $\at{t}$, while $a$ denotes an element of the action set.

    \item ...

    \item $\Tdef$ is called the {\em transition function}. For all $(s,a,s',t) \in \sset \times \aset \times \sset \times \NZ$, let $\T(s,a,s')\coloneqq\Pr(\st{t+1}=s' | \st{t}=s, \ActionT{t}=a)$.
    That is, $\T$ characterizes the distribution over states at time $t+1$ given the state and action at time $t$.

    We allow three alternate notations for $\T$.
    First, let $\TransitionA{s}{a}{s'} \coloneqq \T(s,a,s')$.
    This form takes approximately the same amount of space, but makes it more clear that $\T$ is a conditional distribution over the next state given the current state and action.
    Second, let $\TransitionB{s}{a}{s'} \coloneqq \T(s,a,s')$.
    This notation moves terms into subscripts and superscripts in order to save some space.
    Third, let $\TransitionC{s}{a}{s'} \coloneqq \T(s,a,s')$.
    This final form is particularly useful when space is limited.
    The author is should select one the four notations for $\T$, and remain consistent within each paper.

    \item $\isset$ is the initial distribution of states defined as $\issetdef$.
    \item $\D$ is the discount factor defined as $\Ddef$.

    \item $\pol$ is the policy which is defined as $\poldef$

\end{enumerate}
%
And so on and so forth


\section{Acknowledgments}
\label{sec:thanks}
I want to thank Phillip Thomas for fruitful discussions on the naming and contents of this package.


%-----------------------------------------------------------------------------
%   REFERENCE LIST
%-----------------------------------------------------------------------------

\begin{thebibliography}{1}
\providecommand{\natexlab}[1]{#1}
\providecommand{\url}[1]{\texttt{#1}}
\expandafter\ifx\csname urlstyle\endcsname\relax
  \providecommand{\doi}[1]{doi: #1}\else
  \providecommand{\doi}{doi: \begingroup \urlstyle{rm}\Url}\fi

\bibitem[Sutton and Barto(1998)]{SuttonBarto}
R.~S. Sutton and A.~G. Barto.
\newblock \emph{Reinforcement Learning: {A}n Introduction}.
\newblock MIT Press, Cambridge, MA, 1998.

\end{thebibliography}
\bibliographystyle{abbrvnat}    % Uses author initials (requires abbrvnat.bst)

%-----------------------------------------------------------------------------

\end{document}
